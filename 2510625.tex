\documentclass[12pt]{article}  % 官方要求字号不小于 12 号,此处选择 12 号字体

% 本模板不需要填写年份,以当前电脑时间自动生成
% 请在以下的方括号中填写队伍控制号
\usepackage[UTF8]{ctex}
\usepackage{float}
\usepackage{algorithm}
\usepackage{algorithmic}
\usepackage[2510625]{easymcm}  % 载入 EasyMCM 模板文件
\problem{B}  % 请在此处填写题号
% \usepackage{mathptmx}  % 这是 Times 字体,中规中矩 
\usepackage{mathpazo}  % 这是 COMAP 官方杂志采用的更好看的 Palatino 字体,可替代以上的 mathptmx 宏包

\title{An MCM Paper Made by Team 2510625}  % 标题

% 如需要修改题头(默认为 MCM/ICM),请使用以下命令(此处修改为 MCM)
%\renewcommand{\contest}{MCM}

% 文档开始
\begin{document}

% 此处填写摘要内容
\begin{abstract}
    Here is the abstract of our paper.
    
    \textbf{For Problem 1}, after carefully analyzing the impact of changes in the number of tourists, tax rates, and the proportion of government investment in infrastructure on the final income, environmental quality, and social satisfaction, we performed linear programming and established a hybrid optimization model. This model finds the global optimum from the local optimal solutions and is validated using a population iteration mechanism. Based on past resident satisfaction survey data in Juneau and corresponding tourist numbers for the same years, we conducted a nonlinear fit between social satisfaction and tourist numbers, achieving a reliability of 0.7423. 
    We normalized social satisfaction and environmental quality by multiplying them with appropriate coefficients to facilitate measuring their impact on the objective function. According to the results generated by our model, when the number of tourists is 1,552,632, the tourist tax rate is 8\%, the waste management investment ratio is 0.128, the water resource management investment ratio is 0.113, and the environmental protection investment ratio is 0.159, the total income is $295$ billion, the CO\(_2\) emission is 285,684.21 tons, the resident satisfaction is 65.84 (100 being the maximum), and the environmental index is 0.5014 (1 being the maximum, with lower values being better).
    In the sensitivity analysis, we analyzed the local and global sensitivity of the variables, with the specific results shown in Figure X. From this, we concluded that the number of tourists is the key variable.

    %\textbf{For Problem 1}, 我们仔细分析了旅客人数,税收的变化,财政在基建方面的投入比例对于政府最后的收入,环境质量以及社会满意度所产生的影响后进行了线性规划,建立了混合优化模型从局部最优解找到了全局的最优解,并用种群迭代机制对其进行了验证。基于往年朱诺的居民满意度调查数据和相对应年份的游客数量,我们对社会满意度和游客人数进行了一个非线性的拟合,其可靠度为0.7423。我们通过乘以一个合适的系数,对社会满意度和环境质量进行了去量纲处理,方便衡量对目标函数的影响。根据我们的模型生成的结果显示在游客人数为1552632,游客税率为8\%,废物管理投入占比为0.128,水资源管理投入占比为0.113,环境保护投入占比为0.159的时候,总收入为295000000000美元,$\text{CO}_{2}$的排放量为285684.21吨,居民满意度为65.84(100为最大),环境指数为0.5014(1为最大,越小越好)。敏感性分析上,我们分析了变量的局部敏感性和全局敏感性,具体结果如图x所示并因此得出旅客人数是关键的变量。

    Secondly, that is ...

    Finally, that is ...

    % 美赛论文中无需注明关键字。若您一定要使用,
    % 请将以下两行的注释号 '%' 去除,以使其生效
    % \vspace{5pt}
    % \textbf{Keywords}: MATLAB, mathematics, LaTeX.

\end{abstract}

\maketitle  % 生成 Summary Sheet
\tableofcontents  % 生成目录


% 正文开始
\section{Introduction}
\subsection{Problem Background}
Here is the problem background. Three major problems are discussed in this paper, which are:
\begin{itemize}
    \item \textbf{Geographical Location: }Juneau is the capital of Alaska, located in the southeastern part of the state, with a population of approximately 30,000 residents. 
    \item \textbf{Current Tourism Situation: }In 2023, Juneau set a record for hosting 1.6 million cruise passengers, with up to 7 large cruise ships and 20,000 passengers received in a single day.\cite{1} These tourists brought considerable economic benefits to the city, amounting to approximately \$375 million.\cite{2} However, this rapidly developing tourism industry has also brought a series of problems, especially challenges related to overtourism.
    \item \textbf{Environmental Impact: }Mendenhall Glacier in Juneau is one of the city's main tourist attractions, but in recent years, due to rising temperatures, the glacier has been retreating rapidly. Since 2007, the glacier has retreated a distance equivalent to eight football fields. This environmental change has not only caused damage to the natural landscape but also raised concerns among local residents about the sustainability of tourism.\cite{3}
\end{itemize}

\begin{figure}[H]
	\centering
	\includegraphics[width=.8\textwidth]{glacier.png}
	\caption{Volume Change of Mendenhall Glacier from August 17, 1984 to July 28, 2023}\label{fig:glacier}
\end{figure}

\subsection{Problem Restatement and Analysis}
\begin{itemize}
    \item \textbf{Problem1: }Develop a model for a sustainable tourism industry that should meet the maximization of revenue, the maximization of environmental quality, and the maximization of social satisfaction, and conduct a sensitivity analysis on it.
    \item \textbf{Problem2: }Develop a model to address the issue of visitor diversion, which is also a measure to increase revenue and reduce regional pressure.
    \item \textbf{Problem3: }Demonstrate how the model can be adapted to another tourist destination affected by overtourism, by obtaining relevant information from another city and analyzing it with the model.
    \item \textbf{An article: }Write a memo to the Juneau Convention and Visitors Bureau, outlining the forecast of the results, the impact of various measures, and suggestions on how to optimize the results.
\end{itemize}

\subsection{Our work}
To avoid complicated description , intuitively reflect our work process, the flow chart is show
as the following figure:
\begin{figure}[H]
	\centering
	\includegraphics[width=\textwidth]{Process-map.png}
	\caption{Flowchart of Work}\label{fig:Process-map}
\end{figure}

\section{Preparation of the Models}
\subsection{Assumptions}
对于模型一:
\begin{itemize}
	\item 假设游客消费带来的正面影响会带动就业岗位等间接影响,总体可由游客消费统一衡量。为简化模型,只考虑对旅游业经济影响最重要的因素,即游客消费,不考虑旅游业发展带来的就业岗位等间接经济影响。
	\item 假设税收在一定合理范围内变动时不会对游客消费和游客人数产生剧烈影响。因为提高税收并没有对游客人数增长造成显著的影响,同时考虑到税收增长带来的消费意愿降低和游客消费提高的综合效应。
	\item 为简化模型,假设游客人均消费为常数。由近几年的数据分析,游客人均消费并没有明显波动,对游客人数并没有明显的抑制作用,可认为在对一年内或近几年的较短周期的分析优化时是常量。
	\item 考虑基建压力与其所带来的隐性成本与环境污染,根据居民调查报告与实际情况,选择居民认为压力最大的废物处理系统,与容易出现压力的水供应系统作为隐性成本的代表进行衡量。假设其他如交通压力与能源供应等所受的压力情况与趋势与这两者相似,并由这两者的承载量衡量,在模型中忽略其具体的数学关系。
	\item 假设支出能够直接反馈到承载量与最大碳排放量里面,并且使得其获得线性的增长。为简化模型与评测,综合考虑时间滞后效应下投入对当年的基建等的影响,将其简化为线性函数,并通过调节其前的系数调整支出的即时收益。
\end{itemize}
对于模型二:
\begin{itemize}
	\item 假设宣传的成本消耗与所能分流的人数为线性关系,以便于量化计算宣传成本。
\end{itemize}
\subsection{数据准备}
\begin{tabbing}
    朱诺市近年的年游客数量 \= 朱诺市近年的月游客数量 \= 朱诺市税收分配情况 \kill
    朱诺市近年的年游客数量 \> 朱诺市近年的月游客数量 \> 朱诺市税收分配情况 \\
    朱诺市近年的各项税率 \> 朱诺市近年的游客人均消费 \> 朱诺市近年的游客调查报告 \\
    朱诺市近年的居民调查报告 \> 朱诺市近年的饮用水年供应量 \> 朱诺市近年的日废物处理量 \\
    美国居民日均用水量 \> 美国居民日均废物产生量 \> 九寨沟近年的年游客数量 \\
    九寨沟近年的财政支出 \> 九寨沟近年的游客人均消费 \> 朱诺市观鲸项目项目报告 \\
    朱诺市观鲸项目游客平均消费 \> 朱诺市观景项目近年年游客数量
\end{tabbing}
\subsection{Notations}
The primary notations used in this paper are listed in Table \ref{tb:notation}.

% 三线表示例
\begin{table}[H]
\begin{center}
\caption{Notations}
\begin{tabular}{>{\centering\arraybackslash}m{4cm} >{\centering\arraybackslash}m{10cm}}
	\toprule
		Symbol & Definition \\
	\midrule
		\(R\) & 经济发展指数 \\
	    \(R_{e}\) & Total tourism income \\
		\(N_{t}\) & Number of tourists \\
		\(N_{tmax}\) & Maximum number of tourists allowed per year \\
		\(\tau_{t}\) & Tax rate \\
		\(P_{t}\) & Average spending per tourist \\
		\(E\) & 环境污染指数 \\
		\(CO_{2max}\) & 所允许的最大年 \(CO_{2}\) 排放量 \\
		\(CO_{2p}\) & Carbon emissions per person \\
		\(C_{waste}\) & 城市的废物处理系统能够承载的年游客人数 \\
		\(C_{water}\) & 城市的饮用水供应系统与水污染处理系统能够承载的年游客人数 \\
		\(S_{residents}\) & Residential satisfaction \\
		\(S\) & Social satisfaction \\
		\(P_{waste}\) & Cities' investment in waste management \\
		\(P_{water}\) & Cities' investment in water management \\
		\(P_{e}\) & Cities' investment in environment management \\
		\(Z\) & Representation of target equation\\
	\bottomrule
\end{tabular}\label{tb:notation}
\end{center}
\end{table}

\section{Solution to Problem1}
\subsection{Establish a Sustainable Tourism Development Model}
\subsubsection{Identifying Relationships Between State Variables}
Considering the needs of sustainable tourism development, we evaluate the comprehensive benefits of Juneau's tourism industry from three perspectives: environment, economy, and society. Environmental pollution index $E$, economic index $R$, and social satisfaction index $S$ are set for the evaluation.
%考虑到可持续旅游发展的需要,我们从环境、经济与社会三个角度对朱诺市的旅游业综合收益进行评估。并设置环境污染指数E、经济指数R、与社会满意度指数S来进行评估。

Our objective function should be influenced by income, social satisfaction and environmental quality. We aim to maximize economic effects and social satisfaction while minimizing environmental pollution. Therefore, the economic index $R$ and social satisfaction index $S$ have positive effects on the objective function, while the environmental pollution index has a negative inhibitory effect on the objective function. Finally, we obtain:
%我们的目标函数应受收入,社会满意度和环境质量三者影响,我们希望经济效应最大化,社会满意度最大化以及环境污染最小化。故而,经济指数R与社会满意度指数S对目标函数起到正向的影响作用,而环境污染指数对目标函数有负向的抑制作用。最终我们得到:
\begin{equation}
	Objective\ function:Z=R+S-E
\end{equation}

我们主要通过游客在旅游时的总消费来衡量旅游收入。旅游总收入主要受到人均消费金额与游客总人数的影响。为了简化模型,我们认为在短期内人均消费不会发生很大波动,是定值。故而旅游的总收入应当和旅客的人数成简单的线性关系,因此我们可以很简单的得到公式2
\begin{equation}
    R_{e}(N_{t},P_{t}) = P_{t}N_{t}
\end{equation}

需要注意的是,由于旅游的大部分税收,如酒店税与酒精税均是隐形税收,直接在消费中体现,故而我们认为政府从旅游业中获取的税收是包含在计算出的收入之中的,按照总税收的比例从收入中进行折算获得。

同时,为了实现在目标函数中与环境和社会指数进行数学运算获得目标函数进行收益评估,我们对人均收入进行了归一化处理,消除量纲的差异以便于获得目标函数。
\begin{equation}
    R = R_{e}/R_{estandard}
\end{equation}
其中,$R_{estandard}$是设置的收入的理论最大值。我们认为,由于对环境保护等的需要,较短期内经济不会过大增长,选取略高于最近一年的年收入最高值的常数,作为收入最大值进行归一化处理获得经济指数。

环境质量方面,我们主要考虑三个因素的影响:二氧化碳排放量,水的消耗量以及废物的产生量,而这三个因素都与游客人数密切相关。
对于碳排放因素,我们主要考虑朱诺市由于过度旅游造成的冰川融化问题。根据NASA的数据,门登霍尔冰川在1984-2023年40年间融化了约16公里,平均每年融化40m,需要控制其融化的速度。我们可以使用度日因子法估计冰川的融化速度:
\begin{equation}
    M=DDF\times DDT
\end{equation}
其中,DDF是度日因子,变化小接近常量,而DDT是正积温,受到温度的直接影响。想要控制冰川融化的速度,关键在于控制温度的升高,而碳排放是影响温度升高的主要因素。根据《巴黎协定》的规定,1850-2100年间年平均气温升高尽量控制在1.5度-2.0度,我们以1.5度为温度升高的最大限度计算最大碳排放量。
\begin{equation}
    \Delta T=\lambda\times \ln(CO_{2}/CO_{2pre-industrial})
\end{equation}
根据该公式计算出全球平均二氧化碳浓度增量,乘上大气质量得到全球250年间的二氧化碳排放总量,通过朱诺市的人口占比,估算出朱诺市的年平均排放量。旅游业是朱诺市占比最大的产业,按照总量的百分之60计算排放总量。同时我们根据朱诺市的森林湿地面积与平均碳吸收量,将两者相加得到最大碳排放量。
而对于游客的人均碳足迹,我们考虑到游客主要以邮轮方式出行,有较高的碳排放量,结合邮轮出行平均碳排放量和游客岸上行动的平均碳排放量对人均碳排放量进行了估计。
通过朱诺市对后两个因素的处理能力和游客与居民的人均消耗/产生量,可以计算出朱诺市供水系统和废物处理系统可以每年承载的游客数$C_{waste}$, $C_{water}$,我们将实际的人数和承载人数相比进行归一化处理来衡量基建体系压力。最后结合各因子的重要性因素得到了公式3:
\begin{equation}
	E=k_{1}\frac{CO_{2p}N_{t}}{CO_{2max}}+k_{2}\frac{N_{t}}{C_{waste}}+k_{3}\frac{N_{t}}{C_{water}}
\end{equation}
具体的,我们认为朱诺市目前面临较为严重的冰川融化问题,需要控制碳排放量,同时,其废物处理系统的承载量相比供水系统更小,人数的增长对其的压力更大,故而我们认为k1>k2>k3。

社会满意度方面,由于游客满意度在较长时间跨度的统计数据中均无明显变化,总体满意度停留在百分之99左右,我们认为游客满意度在税收等合理的情况下不会发生明显变化,故而主要考虑居民满意度。
我们主要收集了朱诺市1998,2002,2006,2022,2023五年的社会满意度调查结果\cite{4},摒除其无影响和不知道的无效评价,剩下的评价依据李斯特量表分成了五级,我们按照100分制对数据的评分进行了扩大处理。同时我们查找了这五年的游客人数\cite{5}来和社会满意度进行了数据拟合并得到了如下结果:
\begin{figure}[H]
	\centering
	\includegraphics[width=.8\textwidth]{satisfaction.png}
	\caption{Nonlinear Fitting between $N_t$ and $S_{residents}$}\label{fig:satisfaction}
\end{figure}
其中,$a_1 = -1.9753 \mathrm{e}{-11}$, $a_2 = 4.772\mathrm{e}{-5}$, $b = 39.2660$, $R^2 = 0.7423$。最后,为了消除量纲带来的影响,我们给满意度除以了满意度的最大值100分。得到的公式4如下:
\begin{equation}
	\begin{cases}
		S_{residents}=a_{1}N_{t}^2+a_{2}N_{t} + b_{1} \\
		S=S_{residents}/100
	\end{cases}
\end{equation}

额外投入方面,我们考虑到了政府对于额外税收的分配计划对系统的反馈作用,政府可以向供水系统、废物处理系统以及环境保护方面进行投入,如进行基建建设或者增强植树造林等,以提高基建的最大承载量和最大的碳排放量,进而获取更高的收益。我们认为政府可以调整投入在政府实际分配中的比值,以获取最大的收益。得到公式5如下:
\begin{equation}
	\begin{cases}
		P_{waste} = k_5\tau_tR_e\\
		P_{water} = k_6\tau_tR_e\\
		P_{e}=k_7\tau_{t}R_e \\
		k_5+k_6+k_7 \leqslant 0.4
	\end{cases}
\end{equation}
为了简化模型,我们认为投入能够直接带来回报,并且支出以一定比例直接增加基建承载量或者最大碳排放量,以便于模型做出决策。
\begin{equation}
	\begin{cases}
		C_{waste}+=\alpha_1P_{waste} \\
		C_{water}+=\alpha_2P_{water} \\
		CO_{2max}+=\alpha_3P_e \\
	\end{cases}
	\end{equation}
\subsubsection{找到约束条件}
在经济方面,为了限制税率的值以保证假设准确,我们设置了一个限制,认为总体税率应小于等于8\%。因此有公式7:
\begin{equation}
	Financial:\tau_{t}\leqslant 8\% \\
\end{equation}

游客人数方面,我们依据朱诺市的政策\cite{6}得出了每天的游客的人数限制。公式8如下:
\begin{equation}
	Tourism:
		0\leqslant N_{t}\leqslant N_{tmax}\\
\end{equation}

环境方面,首先,我们限定了二氧化碳的最大排放量,而在水资源最大承载量和废物处理最大承载量上面,我们考虑到游客人数的季节性波动会导致旺季基建压力的上升,根据数据获得旺季每天最多游客人数占全年总游客人数的比例,考虑到基建能力一年内的的承载弹性,升大设置1.2倍日平均承载量作为上限来限制最大的日游客人数。公式9如下:
\begin{equation}
	Environmental:
	\begin{cases}
		N_{t}\cdot CO_{2p}\leqslant CO_{2max} \\
		0.012N_t\leqslant \frac{1.2}{365}C_{waste} \\
		0.012N_t\leqslant \frac{1.2}{365}C_{water}
	\end{cases}
\end{equation}

社会满意度方面,经过合适的量化后,我们将60作为及格线。故公式10如下:
\begin{equation}
	Societal:\ S_{residents} \geqslant 60
\end{equation}	
\subsubsection{Multi-start Optimization Algorithm}
Since our model has five decision variables, namely \(N_t\), \(\tau_t\), \(k_5\), \(k_6\) and \(k_7\), which interact with each other and have varying degree of influence on the objective function. In order to achieve relatively optimal conditions for each decision variable locally and at the same time strive for the global objective to be optimal, such as maximizing tourism revenue, minimizing environmental impact, and maximizing resident satisfaction, etc., we use a multi-start optimization strategy to optimize the various influencing factors of the objective function. The pseudocode is as follows:
\begin{algorithm}
	\caption{Multi-start Optimization Algorithm}
	\begin{algorithmic}[1]
		\STATE \textbf{Input:} $n\_starts$,\ $Nmax$
		\STATE \textbf{Output:} $best\_result$
		\vspace{0.2cm}
		\STATE /* Initialize variables */
		\STATE $best\_result\leftarrow null$
		\STATE $best\_objective\leftarrow \infty$
		\STATE /* Generate starting points for each parameter */
		\STATE $Nt\_starts\leftarrow linspace(100000,\,0.8\times Nmax,\,n\_starts)$
		\STATE $\tau\_starts\leftarrow linspace(0.02,\,0.07,\,n\_starts)$
		\STATE $k\_starts\leftarrow linspace(0.05,\,0.15,\,n\_starts)$
		\vspace{0.2cm}
		\STATE \textbf{Define optimization bounds}
		\vspace{0.2cm}
		\STATE /* Perform optimization from multiple starting points */
		\FOR{$i\leftarrow 0$ \TO $n\_starts-1$}
		\STATE /* Construct initial point */
		\STATE $x0\leftarrow [Nt\_starts[i],\ \tau\_starts[i],\ k\_starts[i],\ k\_starts[i]]$
		\STATE /* Minimize objective function */
		\STATE $result\leftarrow \textbf{minimize\_function}$
		\STATE /* Update best result if better solution found */
		\IF{$result.success$ \AND $result.objective\ <\ best\_objective$}
		\STATE $best\_objective\leftarrow result.objective$
		\STATE $best\_result\leftarrow deepcopy(result)$
		\ENDIF
		\ENDFOR
		\vspace{0.2cm}
		\RETURN $best\_result$
	\end{algorithmic}
\end{algorithm}

\textbf{The solution and output of the model:} The model maximizes economic, environmental, and social benefits through multi-start optimization of five decision variables. In the initialization phase, the algorithm first sets up an empty optimal solution container and an infinite initial optimal objective value as the benchmark. Then, the algorithm generates multiple sets of different starting points within reasonable value ranges to ensure that the algorithm can explore a larger solution space. During the optimization loop phase, the algorithm performs a complete optimization attempt for each set of starting points. Each attempt uses the \textbf{Sequential Least Squares Quadratic Programming} (SLSQP) method to find the solution that optimizes the objective function, considering all constraints, such as visitor number limits, tax rate range, investment ratio restrictions, etc.). If an optimization attempt is successful and the resulting objective function value is better than the currently recorded optimal value, the optimal solution and objective value are updated. This process repeats until all starting points have been tested, and the globally optimal result is retained.

\subsubsection{计算结果}
旅游总收入上,根据以往的数据,我们给旅客平均消费值赋值为x。经过我们的计算可以得到游客每年人数应为x人,$P_e$为x,结合以往的数据来看,这个数据是相对合理的。

居民满意度上,将旅客人数带进拟合方程后可以得到居民满意度为x,符合我们的预期范围。

环境质量上,经过我们的ATP法得出$k1 = X$, $k2 = X$, $k3 = X$,将旅游人数和$CO_{2max}$, $C_{waste}$,$C_{water}$, $N_t$带入方程后得出结果为x,符合我们的预期范围。

\subsection{Sensitivity Analysis}
Since we have used a multi-start optimization strategy, we use \textbf{Sobol Analysis} to simultaneously analyze the local sensitivity and global sensitivity of the input variables $N_t$, $\tau_t$, $k_5$, $k_6$ and $k_7$. The pseudocode is as follows:
\begin{algorithm}
	\caption{Sensitivity Analysis}
	\begin{algorithmic}[1]
		\STATE \textbf{Input:} $n\_samples$
		\STATE \textbf{Output:} $Si$ (result of Sobol Analysis)
		\vspace{0.2cm}
		\STATE /* Define problem structure */
		\STATE \textbf{Initialize problem dictionary:}
		\STATE $\quad$number of variables, variable names, variable bounds
		\STATE /* Generate samples */
		\STATE $param\_values = saltelli.sample(problem,\ n\_samples)$
		\STATE /* Evaluate model for all samples */
		\STATE Initialize empty array $Y$
		\FOR {each parameter set $X$ in $param\_values$}
		\STATE Calculate objective function value for $X$
		\STATE Add result to array $Y$
		\ENDFOR
		\STATE /* Normalize results */
		\STATE $Y = (Y - minimum\ of\ Y) / (maximum\ of\ Y - minimum\ of\ Y)$
		\STATE /* Perform Sobol Analysis */
		\STATE $Si=sobol.analyze(problem,\ Y)$
		\vspace{0.2cm}
		\STATE \textbf{Print results}
		\STATE \textbf{Visualize results}
		\vspace{0.2cm}
		\RETURN $Si$
	\end{algorithmic}
\end{algorithm}

The results are shown in the figure below:
\begin{figure}[H]
	\centering
	\includegraphics[width=.8\textwidth]{sensitive.png}
	\caption{Sensitivity Analysis}\label{fig:sensitivity}
\end{figure}

From the figure, it is clear that the number of tourists is the most critical variable, with its local and global sensitivities reaching $0.877$ and $0.932$, respectively, far greater than the other input variables. This also aligns with the objective fact that, as a tourist city, Juneau's income is heavily influenced by the number of tourists.

\subsection{Model Validation}

在3.1中,我们使用多起点的混合优化策略,计算了我们建立的可持续旅游发展模型,并在3.2中对其进行了敏感度分析。虽然我们的计算结果比较符合实际,但我们仍对其正确性提出了一些质疑。因此,我们又使用了多目标优化策略再次计算了以上模型的最优情况,以验证3.1中多起点的混合优化策略的正确性。我们的优化目标依然是收入最大化、环境影响最小化和社会满意度最大化。在新的多目标优化策略中,我们使用了DEAP(Distributed Evolutionary Algorithms in Python)框架来实现NSGA-II(非支配排序遗传算法II)。然后我们使用帕累托优化,得到该模型下的一系列最优情况,然后根据需求和实际情况,选取最终的结果为 x。这一结果既符合我们的预期,也于3.1中的多起点混合优化策略的结果相互印证。此外,我们同样对多目标优化策略的算法进行了Sobol敏感度分析,得到的结果与3.2中的结果相差无几,这也进一步证明了3.3中的多目标优化策略算法的正确性,以及3.1中多起点的混合优化策略算法的正确性。如图是多目标优化策略算法的执行结果。

\begin{figure}[htbp]
\centering
\begin{subfigure}[b]{.4\textwidth}
\includegraphics[width=\textwidth]{pareto1.png}
\caption{Pareto Front 1}\label{subfig:left1}
\end{subfigure}
\begin{subfigure}[b]{.4\textwidth}
	\includegraphics[width=\textwidth]{pareto2.png}
	\caption{Pareto Front 2}\label{subfig:right1}
\end{subfigure}
\end{figure}

\begin{figure}[htbp]
	\centering
	\begin{subfigure}[b]{.4\textwidth}
		\includegraphics[width=\textwidth]{pareto3.png}
		\caption{Pareto Front 3}\label{subfig:left2}
	\end{subfigure}
	\begin{subfigure}[b]{.4\textwidth}
		\includegraphics[width=\textwidth]{pareto4.png}
		\caption{Pareto Front 4}\label{subfig:right2}
	\end{subfigure}
\end{figure}

\section{Solution to Problem2}
为了展示模型是如何适应另一个受过度旅游影响的,我们经过数据的查找和收集决定选择位于中国四川省的九寨沟风景区来进行分析。
\subsection{数据收集}
因为每个地方的基础设施压力,人均消费等都大不相同,我们需要经过数据调研修改相应的参数后才能使用当前的模型进行预测。

通过对官网数据的收集\cite{8},我们可以得出九寨沟县每年的收入和九寨沟风景区之间的关系。但是由于官方网站缺乏相应的满意度调查等信息,我们根据社会上的普遍评价对这些值进行了简单的估计,通过修改相应的限制条件,如水资源的人均消耗量,废物的人均产生量,水资源和废物处理的承载能力,游客人均消费水平,我们可以将模型适用于九寨沟的情况。
\begin{figure}[H]
	\centering
	\includegraphics[width=.8\textwidth]{JiuZhaiGou.jpg}
	\caption{The Relationship between Tourism Revenue and Number of Visitors in Jiuzhaigou Scenic Area}\label{fig:JiuZhaiGou}
\end{figure}
\subsection{结果展示}
根据我们模型跑出来的结果显示,xxx。这些数据和往年相比具有合理性,符合实际的基本预期。


\section{Solution to Problem3}
\subsection{生成不同景点的目标函数}
为了进行不同景点之间分流,我们需要获取各个景点的目标函数以来衡量各景点人流量变化后的收益变化。因此,我们需要获得具体每个景点的相关数据。因为不同景点的游客人数,人均消费不同,随之而导致的基建压力和社会满意度也大不相同,所以我们需要修改模型一的参数以来适应不同景点的实际情况。

阐述一下不同景点修改的实际数据
\subsection{建立分流模型}
在对景点进行了目标函数的构建之后,我们通过各景点的经过修改后的模型一计算得出具体的最优游客人数和各景点的支出分配政策。

在此基础上,我们考虑分流问题。假设模型中有$n$个子模型,代表了具体的$n$个景点。则总收益函数为:
\begin{equation}
    Z = \sum_{i=1}^{n} Z_i
\end{equation}

每个模型中的实际人数为$N_{ti}$,,最优人数为$N_{ei}$。对于实际人数大于最优人数的景点$n_i$,我们希望将超出最有人数的部分分流去$n_j$,则
$C_{i-j} = C_{pi-j}*\Delta N_{i-j}$(总成本,等于将一个人分流走的宣传成本乘以分流走的人数),则宣传需要$C = \sum_{i,j} C_{i-j}$,
故决策变量为:
\begin{equation}
    \Delta N_{i,j} (1 \leq i,j \leq n)
\end{equation}
目标函数为:
\begin{equation}
	Z = \sum_{i=1}^{n} Z_i(N_{ti}) - \frac{C}{C_{std}}
\end{equation}
\subsection{结果展示}
这里以朱诺市两个景点门登霍尔冰川和赏鲸雨林为例来具体展示分流模型的结果。
\begin{figure}[H]
	\centering
	\includegraphics[width=.8\textwidth]{model2.png}
	\caption{model2}\label{fig:model2}
\end{figure}
解释一下这个图,我看不太懂QAQ

% 以下为信件/备忘录部分,不需要可自行去掉
% 如有需要可将整个 letter 环境移动到文章开头或中间
% 请在第二个花括号内填写标题,如「信件」(Letter)或「备忘录」(Memorandum)
\begin{letter}{Memorandum}
\begin{flushleft}  % 左对齐环境,无首行缩进
\textbf{To:} 朱诺市政府\\
\textbf{From:} Team 2510625\\
\textbf{Date:} January 27th, 2025\\
\textbf{Subject:} 关于政府在旅游业上的一些建议
\end{flushleft}

随着旅游市场的不断发展,朱诺市作为阿拉斯加的首府,拥有丰富的自然景观和独特的文化魅力,吸引了大量游客前来观光旅游。为了更好地了解旅游市场的发展趋势,评估各种措施对旅游市场的影响,并提出优化建议,我们进行了深入的市场调研和数据分析。

首先是关于旅游市场的预测,根据我们的市场调研和数据分析,预计未来几年朱诺市的旅游市场将呈现以下趋势:
\textbf{游客数量增长:}随着全球经济的复苏和人们生活水平的提高,预计未来几年朱诺市的游客数量将呈现稳步增长的趋势。特别是来自亚洲和欧洲的游客数量将有较大幅度的增加。
\textbf{旅游季节变化:}虽然夏季仍然是朱诺市的旅游旺季,但随着冬季旅游项目的不断丰富,如狗拉雪橇、冰钓等,冬季旅游市场也将逐渐升温。

二是各种措施的影响,为了促进朱诺市旅游市场的发展,我们分析了一系列措施,这些措施对旅游市场产生了以下影响:
\textbf{适量提高税率:}适量提高税率短时间不会对旅游人数产生较大的影响,而税率的提高可以增加旅游业带来的收入,额外收入增加后可以对基建投入更多的金钱以来提高水资源和废物处理承载能力。
\textbf{限制每日的游客人数:}限制每日的人数有助于缓解基础设施压力,提高居民满意度,同时如果限制的人数控制在一个比奥较好的值也能确保每日的收入保持在一个比较高的水平。

最后是一些优化的建议:
\textbf{加强旅游基础设施建设:}继续加大对旅游基础设施的投入,进一步改善住宿、餐饮、交通等方面的条件。特别是要加大对偏远地区的旅游基础设施建设,提高旅游的可达性和便利性。有利于游客参观更多的景点,分散个别热门景点的压力。
\textbf{提升旅游服务质量:}加强对旅游从业人员的培训,提高旅游服务的质量和水平。特别是要加强对导游、酒店服务人员等的培训,提高他们的专业素质和服务意识。这样可以提升朱诺市的旅游风评,增加游客前来的意愿。
\textbf{加强市场营销推广:}继续在国内外主要旅游市场进行市场营销推广,提高朱诺市的知名度和美誉度。可以通过举办旅游推介会、参加旅游展会等方式,向更多的游客宣传朱诺市的旅游资源和旅游产品以来吸引更多游客和刺激游客消费,提高游客平均消费水平。

\end{letter}
\clearpage

% 参考文献,此处以 MLA 引用格式为例

\begin{thebibliography}{99}
\bibitem{1} \url{https://abc7.com/post/juneau-alaska-cruise-ship-limits-overtourism/15048713/}
\bibitem{2} \url{https://juneau.org/wp-content/uploads/2024/01/CBJ-Cruise-Impacts-2023-Report-1.22.24.pdf}
\bibitem{3} \url{https://alaskapublic.org/2023/08/07/crammed-with-tourists-juneau-wonders-what-will-happen-as-mendenhall-glacier-recedes/}
\bibitem{4} \emph{A simple, easy \LaTeX\ template for MCM/ICM: EasyMCM}. (2018). Retrieved December 1, 2019, from\url{https://www.cnblogs.com/xjtu-blacksmith/p/easymcm.html}
\bibitem{7} \url{https://earthobservatory.nasa.gov/images/151682/alaskas-mendenhall-glacier}
\bibitem{8} \url{https://www.jiuzhai.com/news/number-of-tourists/}
\end{thebibliography}


% 以下为附录内容
% 如您的论文中不需要附录,请自行删除
\begin{subappendices}  % 附录环境

\section{Appendix A: 关于AI工具的使用说明}
To clarify the importance of using \LaTeX\ in MCM or ICM, several points need to be covered, which are \ldots

To be more specific, \ldots

All in all, \ldots

Anyway, nobody \textbf{really} needs such appendix \ldots

\end{subappendices}  % 附录内容结束

\end{document}  % 结束
